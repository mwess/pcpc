% --------------------------------------------------------------
% This is all preamble stuff that you don't have to worry about.
% Head down to where it says "Start here"
% --------------------------------------------------------------
 
\documentclass[12pt]{scrartcl}
 
\usepackage[margin=1in]{geometry} 
\usepackage{amsmath,amsthm,amssymb}
\usepackage[utf8]{inputenc}
\usepackage{scrpage2}
 
\newcommand{\N}{\mathbb{N}}
\newcommand{\Z}{\mathbb{Z}}
 
\newenvironment{theorem}[2][Theorem]{\begin{trivlist}
\item[\hskip \labelsep {\bfseries #1}\hskip \labelsep {\bfseries #2.}]}{\end{trivlist}}
\newenvironment{lemma}[2][Lemma]{\begin{trivlist}
\item[\hskip \labelsep {\bfseries #1}\hskip \labelsep {\bfseries #2.}]}{\end{trivlist}}
\newenvironment{exercise}[2][Exercise]{\begin{trivlist}
\item[\hskip \labelsep {\bfseries #1}\hskip \labelsep {\bfseries #2.}]}{\end{trivlist}}
\newenvironment{reflection}[2][Reflection]{\begin{trivlist}
\item[\hskip \labelsep {\bfseries #1}\hskip \labelsep {\bfseries #2.}]}{\end{trivlist}}
\newenvironment{proposition}[2][Proposition]{\begin{trivlist}
\item[\hskip \labelsep {\bfseries #1}\hskip \labelsep {\bfseries #2.}]}{\end{trivlist}}
\newenvironment{corollary}[2][Corollary]{\begin{trivlist}
\item[\hskip \labelsep {\bfseries #1}\hskip \labelsep {\bfseries #2.}]}{\end{trivlist}}
 
\begin{document}
 
% --------------------------------------------------------------
%                         Start here
% --------------------------------------------------------------
 
%\renewcommand{\qedsymbol}{\filledbox}

\pagestyle{scrheadings}
\clearscrheadfoot
\ohead{Maximilian Wess, Hanna Holderied}
 
\title{Practical Course in Parallel Computing - Exercise 9}%replace X with the appropriate number
\author{Maximilian Wess, Hanna Holderied\\
\tiny{last edited \today}}
\date{}

\maketitle
%\newpage

\begin{enumerate}
    \item{Exercise 1}
        \begin{enumerate}
            \item For Exercise 1, please see e1.c. To solve the problem lines 59 and 60 need to be switched.
            \item Using for instance, mpirun -n 2 -hosts c000,c001 ./e1
            \item See e1\_pbs.sh
        \end{enumerate}
    \item{Exercise 2}
        \begin{enumerate}
            \item See e21.c.
            \item See e22.c
        \end{enumerate}
    \item{Exercise 3}
        Using e3c. the functions can be explained as follows( not that e3 was executed with mpirun -n 4 e3):
        \begin{enumerate}
            \item{MPI\_Gather} Arrays x and y are initialized with whitespaces. The first position of x is filled with 'a' + rank. The root rank in this case is 1. Using MPI\_Gather all processes send the first character in the x array to the root process, which uses the char array y as a receive buffer. For that reason the y array contains a b c d. Moreover MPI\_Gather orders the received data chunks, i.e. process 0 gets the first part of the receiving array, pprocess 1 gets the second part and so on. 
            \item{MPI\_Allgather} MPI\_Allgather workds similiar to MPI\_Gather except that not only the root process, but all processes receive the send buffer.
            \item{MPI\_Scatter} The send buffer of the root process is partitioned into chunks. TGhen each chunk is sent to one process. The receivers are again ordered, i.e. process 0 gets the first chunks, process 1 the second and so on. In the example one can see that the array e f g h is partitioned into 4 chunks. Process 0 receives e, process 1 receives f,...
            \item{MPI\_Alltoall} In each process the sendbuff is split into n chunks (n is the number of processes). Each chunk contains sendcount elements. The recvbuf is partitioned accordingly. The process j sends the k-th block of its local sendbuff to process k, which places the data in the j-th block of its recvbyuf. In the exapmle this means for instance, that for n=4 chunk 0 of process 3 (m) is send to process 0 and is placed in the recv buffer at position 3. 
            \item{MPI\_Bcast} The root process sends the send buffer to every other process. In the example only the root process has character b in the sendbufgf, but after calling MPI\_Bcast every other process has received the character.
        \end{enumerate}
    \item{Exercise 4}
        \begin{enumerate}
            \item See ex1\_plot.pdf.
            \item See e4b.c and ex2\_plot.pdf.
            \item See e4c.c and ex3\_plot.pdf.
            \item See e4d1.c, e4d2.c, e4d3.c and e4d4.c.
        \end{enumerate}
\end{enumerate}


%% --------------------------------------------------------------
% This is all preamble stuff that you don't have to worry about.
% Head down to where it says "Start here"
% --------------------------------------------------------------
 
\documentclass[12pt]{scrartcl}
 
\usepackage[margin=1in]{geometry} 
\usepackage{amsmath,amsthm,amssymb}
\usepackage[utf8]{inputenc}
\usepackage{scrpage2}
 
\newcommand{\N}{\mathbb{N}}
\newcommand{\Z}{\mathbb{Z}}
 
\newenvironment{theorem}[2][Theorem]{\begin{trivlist}
\item[\hskip \labelsep {\bfseries #1}\hskip \labelsep {\bfseries #2.}]}{\end{trivlist}}
\newenvironment{lemma}[2][Lemma]{\begin{trivlist}
\item[\hskip \labelsep {\bfseries #1}\hskip \labelsep {\bfseries #2.}]}{\end{trivlist}}
\newenvironment{exercise}[2][Exercise]{\begin{trivlist}
\item[\hskip \labelsep {\bfseries #1}\hskip \labelsep {\bfseries #2.}]}{\end{trivlist}}
\newenvironment{reflection}[2][Reflection]{\begin{trivlist}
\item[\hskip \labelsep {\bfseries #1}\hskip \labelsep {\bfseries #2.}]}{\end{trivlist}}
\newenvironment{proposition}[2][Proposition]{\begin{trivlist}
\item[\hskip \labelsep {\bfseries #1}\hskip \labelsep {\bfseries #2.}]}{\end{trivlist}}
\newenvironment{corollary}[2][Corollary]{\begin{trivlist}
\item[\hskip \labelsep {\bfseries #1}\hskip \labelsep {\bfseries #2.}]}{\end{trivlist}}
 
\begin{document}
 
% --------------------------------------------------------------
%                         Start here
% --------------------------------------------------------------
 
%\renewcommand{\qedsymbol}{\filledbox}

\pagestyle{scrheadings}
\clearscrheadfoot
\ohead{Maximilian Wess, Hanna Holderied}
 
\title{Practical Course in Parallel Computing - Exercise 9}%replace X with the appropriate number
\author{Maximilian Wess, Hanna Holderied\\
\tiny{last edited \today}}
\date{}

\maketitle
%\newpage

\begin{enumerate}
    \item{Exercise 1}
        \begin{enumerate}
            \item For Exercise 1, please see e1.c. To solve the problem lines 59 and 60 need to be switched.
            \item Using for instance, mpirun -n 2 -hosts c000,c001 ./e1
            \item See e1\_pbs.sh
        \end{enumerate}
    \item{Exercise 2}
        \begin{enumerate}
            \item See e21.c.
            \item See e22.c
        \end{enumerate}
    \item{Exercise 3}
        Using e3c. the functions can be explained as follows( not that e3 was executed with mpirun -n 4 e3):
        \begin{enumerate}
            \item{MPI\_Gather} Arrays x and y are initialized with whitespaces. The first position of x is filled with 'a' + rank. The root rank in this case is 1. Using MPI\_Gather all processes send the first character in the x array to the root process, which uses the char array y as a receive buffer. For that reason the y array contains a b c d. Moreover MPI\_Gather orders the received data chunks, i.e. process 0 gets the first part of the receiving array, pprocess 1 gets the second part and so on. 
            \item{MPI\_Allgather} MPI\_Allgather workds similiar to MPI\_Gather except that not only the root process, but all processes receive the send buffer.
            \item{MPI\_Scatter} The send buffer of the root process is partitioned into chunks. TGhen each chunk is sent to one process. The receivers are again ordered, i.e. process 0 gets the first chunks, process 1 the second and so on. In the example one can see that the array e f g h is partitioned into 4 chunks. Process 0 receives e, process 1 receives f,...
            \item{MPI\_Alltoall} In each process the sendbuff is split into n chunks (n is the number of processes). Each chunk contains sendcount elements. The recvbuf is partitioned accordingly. The process j sends the k-th block of its local sendbuff to process k, which places the data in the j-th block of its recvbyuf. In the exapmle this means for instance, that for n=4 chunk 0 of process 3 (m) is send to process 0 and is placed in the recv buffer at position 3. 
            \item{MPI\_Bcast} The root process sends the send buffer to every other process. In the example only the root process has character b in the sendbufgf, but after calling MPI\_Bcast every other process has received the character.
        \end{enumerate}
    \item{Exercise 4}
        \begin{enumerate}
            \item See ex1\_plot.pdf.
            \item See e4b.c and ex2\_plot.pdf.
            \item See e4c.c and ex3\_plot.pdf.
            \item See e4d1.c, e4d2.c, e4d3.c and e4d4.c.
        \end{enumerate}
\end{enumerate}


%% --------------------------------------------------------------
% This is all preamble stuff that you don't have to worry about.
% Head down to where it says "Start here"
% --------------------------------------------------------------
 
\documentclass[12pt]{scrartcl}
 
\usepackage[margin=1in]{geometry} 
\usepackage{amsmath,amsthm,amssymb}
\usepackage[utf8]{inputenc}
\usepackage{scrpage2}
 
\newcommand{\N}{\mathbb{N}}
\newcommand{\Z}{\mathbb{Z}}
 
\newenvironment{theorem}[2][Theorem]{\begin{trivlist}
\item[\hskip \labelsep {\bfseries #1}\hskip \labelsep {\bfseries #2.}]}{\end{trivlist}}
\newenvironment{lemma}[2][Lemma]{\begin{trivlist}
\item[\hskip \labelsep {\bfseries #1}\hskip \labelsep {\bfseries #2.}]}{\end{trivlist}}
\newenvironment{exercise}[2][Exercise]{\begin{trivlist}
\item[\hskip \labelsep {\bfseries #1}\hskip \labelsep {\bfseries #2.}]}{\end{trivlist}}
\newenvironment{reflection}[2][Reflection]{\begin{trivlist}
\item[\hskip \labelsep {\bfseries #1}\hskip \labelsep {\bfseries #2.}]}{\end{trivlist}}
\newenvironment{proposition}[2][Proposition]{\begin{trivlist}
\item[\hskip \labelsep {\bfseries #1}\hskip \labelsep {\bfseries #2.}]}{\end{trivlist}}
\newenvironment{corollary}[2][Corollary]{\begin{trivlist}
\item[\hskip \labelsep {\bfseries #1}\hskip \labelsep {\bfseries #2.}]}{\end{trivlist}}
 
\begin{document}
 
% --------------------------------------------------------------
%                         Start here
% --------------------------------------------------------------
 
%\renewcommand{\qedsymbol}{\filledbox}

\pagestyle{scrheadings}
\clearscrheadfoot
\ohead{Maximilian Wess, Hanna Holderied}
 
\title{Practical Course in Parallel Computing - Exercise 9}%replace X with the appropriate number
\author{Maximilian Wess, Hanna Holderied\\
\tiny{last edited \today}}
\date{}

\maketitle
%\newpage

\begin{enumerate}
    \item{Exercise 1}
        \begin{enumerate}
            \item For Exercise 1, please see e1.c. To solve the problem lines 59 and 60 need to be switched.
            \item Using for instance, mpirun -n 2 -hosts c000,c001 ./e1
            \item See e1\_pbs.sh
        \end{enumerate}
    \item{Exercise 2}
        \begin{enumerate}
            \item See e21.c.
            \item See e22.c
        \end{enumerate}
    \item{Exercise 3}
        Using e3c. the functions can be explained as follows( not that e3 was executed with mpirun -n 4 e3):
        \begin{enumerate}
            \item{MPI\_Gather} Arrays x and y are initialized with whitespaces. The first position of x is filled with 'a' + rank. The root rank in this case is 1. Using MPI\_Gather all processes send the first character in the x array to the root process, which uses the char array y as a receive buffer. For that reason the y array contains a b c d. Moreover MPI\_Gather orders the received data chunks, i.e. process 0 gets the first part of the receiving array, pprocess 1 gets the second part and so on. 
            \item{MPI\_Allgather} MPI\_Allgather workds similiar to MPI\_Gather except that not only the root process, but all processes receive the send buffer.
            \item{MPI\_Scatter} The send buffer of the root process is partitioned into chunks. TGhen each chunk is sent to one process. The receivers are again ordered, i.e. process 0 gets the first chunks, process 1 the second and so on. In the example one can see that the array e f g h is partitioned into 4 chunks. Process 0 receives e, process 1 receives f,...
            \item{MPI\_Alltoall} In each process the sendbuff is split into n chunks (n is the number of processes). Each chunk contains sendcount elements. The recvbuf is partitioned accordingly. The process j sends the k-th block of its local sendbuff to process k, which places the data in the j-th block of its recvbyuf. In the exapmle this means for instance, that for n=4 chunk 0 of process 3 (m) is send to process 0 and is placed in the recv buffer at position 3. 
            \item{MPI\_Bcast} The root process sends the send buffer to every other process. In the example only the root process has character b in the sendbufgf, but after calling MPI\_Bcast every other process has received the character.
        \end{enumerate}
    \item{Exercise 4}
        \begin{enumerate}
            \item See ex1\_plot.pdf.
            \item See e4b.c and ex2\_plot.pdf.
            \item See e4c.c and ex3\_plot.pdf.
            \item See e4d1.c, e4d2.c, e4d3.c and e4d4.c.
        \end{enumerate}
\end{enumerate}


%% --------------------------------------------------------------
% This is all preamble stuff that you don't have to worry about.
% Head down to where it says "Start here"
% --------------------------------------------------------------
 
\documentclass[12pt]{scrartcl}
 
\usepackage[margin=1in]{geometry} 
\usepackage{amsmath,amsthm,amssymb}
\usepackage[utf8]{inputenc}
\usepackage{scrpage2}
 
\newcommand{\N}{\mathbb{N}}
\newcommand{\Z}{\mathbb{Z}}
 
\newenvironment{theorem}[2][Theorem]{\begin{trivlist}
\item[\hskip \labelsep {\bfseries #1}\hskip \labelsep {\bfseries #2.}]}{\end{trivlist}}
\newenvironment{lemma}[2][Lemma]{\begin{trivlist}
\item[\hskip \labelsep {\bfseries #1}\hskip \labelsep {\bfseries #2.}]}{\end{trivlist}}
\newenvironment{exercise}[2][Exercise]{\begin{trivlist}
\item[\hskip \labelsep {\bfseries #1}\hskip \labelsep {\bfseries #2.}]}{\end{trivlist}}
\newenvironment{reflection}[2][Reflection]{\begin{trivlist}
\item[\hskip \labelsep {\bfseries #1}\hskip \labelsep {\bfseries #2.}]}{\end{trivlist}}
\newenvironment{proposition}[2][Proposition]{\begin{trivlist}
\item[\hskip \labelsep {\bfseries #1}\hskip \labelsep {\bfseries #2.}]}{\end{trivlist}}
\newenvironment{corollary}[2][Corollary]{\begin{trivlist}
\item[\hskip \labelsep {\bfseries #1}\hskip \labelsep {\bfseries #2.}]}{\end{trivlist}}
 
\begin{document}
 
% --------------------------------------------------------------
%                         Start here
% --------------------------------------------------------------
 
%\renewcommand{\qedsymbol}{\filledbox}

\pagestyle{scrheadings}
\clearscrheadfoot
\ohead{Maximilian Wess, Hanna Holderied}
 
\title{Practical Course in Parallel Computing - Exercise 9}%replace X with the appropriate number
\author{Maximilian Wess, Hanna Holderied\\
\tiny{last edited \today}}
\date{}

\maketitle
%\newpage

\begin{enumerate}
    \item{Exercise 1}
        \begin{enumerate}
            \item For Exercise 1, please see e1.c. To solve the problem lines 59 and 60 need to be switched.
            \item Using for instance, mpirun -n 2 -hosts c000,c001 ./e1
            \item See e1\_pbs.sh
        \end{enumerate}
    \item{Exercise 2}
        \begin{enumerate}
            \item See e21.c.
            \item See e22.c
        \end{enumerate}
    \item{Exercise 3}
        Using e3c. the functions can be explained as follows( not that e3 was executed with mpirun -n 4 e3):
        \begin{enumerate}
            \item{MPI\_Gather} Arrays x and y are initialized with whitespaces. The first position of x is filled with 'a' + rank. The root rank in this case is 1. Using MPI\_Gather all processes send the first character in the x array to the root process, which uses the char array y as a receive buffer. For that reason the y array contains a b c d. Moreover MPI\_Gather orders the received data chunks, i.e. process 0 gets the first part of the receiving array, pprocess 1 gets the second part and so on. 
            \item{MPI\_Allgather} MPI\_Allgather workds similiar to MPI\_Gather except that not only the root process, but all processes receive the send buffer.
            \item{MPI\_Scatter} The send buffer of the root process is partitioned into chunks. TGhen each chunk is sent to one process. The receivers are again ordered, i.e. process 0 gets the first chunks, process 1 the second and so on. In the example one can see that the array e f g h is partitioned into 4 chunks. Process 0 receives e, process 1 receives f,...
            \item{MPI\_Alltoall} In each process the sendbuff is split into n chunks (n is the number of processes). Each chunk contains sendcount elements. The recvbuf is partitioned accordingly. The process j sends the k-th block of its local sendbuff to process k, which places the data in the j-th block of its recvbyuf. In the exapmle this means for instance, that for n=4 chunk 0 of process 3 (m) is send to process 0 and is placed in the recv buffer at position 3. 
            \item{MPI\_Bcast} The root process sends the send buffer to every other process. In the example only the root process has character b in the sendbufgf, but after calling MPI\_Bcast every other process has received the character.
        \end{enumerate}
    \item{Exercise 4}
        \begin{enumerate}
            \item See ex1\_plot.pdf.
            \item See e4b.c and ex2\_plot.pdf.
            \item See e4c.c and ex3\_plot.pdf.
            \item See e4d1.c, e4d2.c, e4d3.c and e4d4.c.
        \end{enumerate}
\end{enumerate}


%\input{ex01/main}

\end{document}


\end{document}


\end{document}


\end{document}
